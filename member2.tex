% Thoroughly study the following paper by Koushik Sen on using dynamic testing
% to establish whether an alarm is a real data race or not: K. Sen, "Race
% Directed Random Testing of Concurrent Programs," in Proc. ACM SIGPLAN
% Conference on Programming Language Design and Implementation (PLDI'08),
% 2008, pp. 11-21.  http://srl.cs.berkeley.edu/~ksen/papers/raced.pdf

% Describe in detail the technique introduced in this paper. Make sure to
% critically assess its pros and cons. In addition, connect this work to your
% own research. Could you envision it being used in your own work on OpenMP
% data race detection? If so, then describe how. If not, then clarify why
% not. Your response should be roughly 3-4 pages long. Good luck.

\begin{refsection}
\section{Questions from Zvonimir Rakamari\'c}
\label{sec:member2}

\textbf{Describe in detail the technique introduced
  in~\cite{Sen:2008:RDR:1375581.1375584}.
  %
  Make sure to critically assess its pros and cons.
  %
  In addition, connect this work to your own research.
  %
  Could you envision it being used in your own work on OpenMP data race
  detection?
  %
  If so, then describe how.
  %
  If not, then clarify why not.}

\vspace{10pt}
\noindent
In the paper~\cite{Sen:2008:RDR:1375581.1375584} the author proposes a
technique to establish whether a reported race in a concurrent program is
harmful or not.
%
This technique is called \emph{race--directed random testing} or \rfuz.
%
A harmful race is defined as a race that could lead to errors or exceptions
in the programs (i.e.\ crash) such as run-time errors or programmer defined
assertions.
%
Finding these bugs caused by data races is hard.
%
Existing techniques are either imprecise (e.g.\ static, dynamic and hybrid
race detectors), and reported bugs require manual inspection to determine
whether they are real races or just false alarms.
%
On the contrary, precise techniques cannot predict potential races that could
happen in other paths of the program (e.g.\ happens-before based
detectors~\cite{Flanagan:2009}).
\\

\noindent
The \rfuz technique consists of two phases:

\paragraph{Phase 1 - Hybrid-Race Detection:}
The phase~1 of the algorithm is quite simple, the tool the author implemented
uses an existing hybrid race detection
technique~\cite{O'Callahan:2003:HDD:781498.781528}, which combines
lockset-based detection and happens-before-based detection to identify and
create a set of potential races in a program.
%
Each race in the set is identified as a pair of concurrent statements, called
\emph{racing pair of statements}, that can potentially race during a
concurrent execution.

From the paper:

\noindent\rule[0.5ex]{\linewidth}{1pt}

At runtime the race detection algorithm checks, for each pair of events
$<e_i,e_j>$, the following conditions:

\begin{enumerate}
\item
  $e_i = MEM(\sigma_i, m_i, a_i, t_i, L_i) \wedge e_j = MEM(\sigma_j, m_j,
  a_j, t_j, L_j)$,
  where $MEM(\sigma, m, a, t, L)$ denotes that a thread $t$ performs an access
  $a$ (read or write) to a memory location $m$ while holding the set of locks
  $L$ and executing the statement $\sigma$;
\item $t_i \neq t_j$, the two events are performed by two different
  threads;
\item $m_i = m_j$, the two events access the same memory location;
\item $a_i = WRITE \vee a_j = WRITE$, at least one of the memory accesses is a
  write;
\item \label{itm:lockset} $L_i \cap L_j = \emptyset$, the two events are not holding a common lock;
\item $\neg(e_i \prec e_j) \wedge \neg(e_j \prec e_i)$, one of the events (and
  so the memory accesses) does not happens-before the other (i.e.\ they are
  concurrent).
\end{enumerate}

If these conditions hold, then $(\sigma_i, \sigma_j)$ is a racing pair of
statements.
%
All the detected racing pairs will form the $RaceSet$ that will be used by the
phase~2 of the \rfuz.

\noindent\rule[0.5ex]{\linewidth}{1pt}

Phase~1 applies a hybrid race detection
technique~\cite{O'Callahan:2003:HDD:781498.781528} that combines the lock-set
detection and the happens-before approaches.
%
The hybrid technique essentially merges the lockset detection check
(condition~\ref{itm:lockset}) and a limited happens-before detection check.
%
The happens-before approach reports real races, and they are a subset of the
races reported by lockset-based detection, as stated by Theorem~1
in~\cite{O'Callahan:2003:HDD:781498.781528}.
%
Some of the details are intentionally omitted in order to focus on the novel
contribution of \rfuz which is discussed the phase~2.

\paragraph{Phase 2 - \rfuz:}
The phase~2 is the core of the \rfuz algorithm.
%
\rfuz will try to determine if each racing pair in the $RaceSet$ will actually
race in a real concurrent execution~\footnote{Due to imprecisions of the
  hybrid-race detection technique it might not actually race.}.
%
\rfuz re-executes the program with a random schedule for each pair in the set
of potential race.
%
The random scheduling execution increases the probability for \rfuz to find
harmful races, as it can enforce a certain order between two statements.
%
This is different to normal random testing which may always propose always the
same order of statements.
%
% \rfuz will try to determine if each racing pair of statements
% $(\sigma_i, \sigma_j) \in RaceSet$ will actually race in a real concurrent
% execution~\footnote{Due to imprecision of the hybrid-race detection technique
%   it might not actually race}.
% %
% \rfuz re-executes the program with a random schedule for each pair in the set
% of potential race $<\sigma_1,\sigma_2>$ (with $<\sigma_1,\sigma_2>$ the pair
% of concurrent statements).
% %
% During the execution \rfuz picks a random thread at each program state.
% %
% When the thread next-statement $ns_1$ is in $<\sigma_1,\sigma_2>$, postpone
% its execution.
% %
% \rfuz picks another thread and check if next-statement $ns_2$ is in
% $<\sigma_1,\sigma_2>$ and forms a race with $ns_1$~\footnote{Two statements
%   race when they are executed by two different threads, both access the same
%   memory location, and at least one of the statement is a write.}.
% %
% If this happens, \rfuz has found a real race formed by $<ns_1,ns_2>$.
% %
% When \rfuz reports a race, in order to continue the algorithm, it executes
% randomly one of the racing statements and keeps to postpone the execution of the
% other one.
%
% The random scheduling execution increases the probability for \rfuz to find
% harmful races, because it can enforce a certain order between two statements,
% differently than a normal random testing that may propose always the same
% order of statements.
%
% (for example in Figure~1
% of~\cite{Sen:2008:RDR:1375581.1375584} a false race would be reported by
% a hybrid-race detection technique while there is no race since the two
% instructions order is ensured by data)
\\

\noindent
Before going into the details of the \rfuz algorithm, let us consider the
following definitions:

\begin{itemize}
\item $Enabled(s)$ indicates the set of threads that are enabled in the
  program state $s$.
\item $Alive(s)$ indicates the set of threads that are still running and have
  not yet reached the state $s$.
\item $Execute(s,t)$ returns the state reached by executing the next statement
  of thread $t$ in state $s$.
\item $NextStmt(s,t)$ indicates the next statement that thread $t$ would
  execute in state $s$.
\item $Racing(s,t,postponed)$ returns the thread $t' \in postponed$ such that
  the statements that $t$ and $t'$ are about to execute (respectively
  $NextStmt(s,t)$ and $NextStmt(s,t')$) access the same memory location and at
  least one of the accesses is a write (race).
\end{itemize}

\noindent
Therefore, given in input the initial state of the program $s_0$ and the
$RaceSet$, the \rfuz algorithm (Algorithm~1 in
\cite{Sen:2008:RDR:1375581.1375584}) for a given racing pair performs the
following steps:

\begin{itemize}
\item Initialize the current program state $s$ to $s_0$.
\item Initialize an empty set to keep the statements where execution will be
  postponed ($postponed = \emptyset$).
\item As long as there are $Enabled$ threads in the current state $s$, \rfuz
  will keep executing the following steps:
  \begin{enumerate}
  \item Pick a random thread $t$ from $Enabled(s)$ that are not in the
    $postponed$ set (this assures to pick a different thread every time).
  \item Whenever a thread is about to execute one of the racing statements
    ($NextStmt(s,t) \in RaceSet$) the scheduler \emph{postpones} it
    ($postponed := postponed \cup {t}$).
  \item The execution remains postponed, until a different other thread also
    tries to execute one of the racing statements: if the two statements
    access the same memory location and one of them is a write, \rfuz has
    found a race ($Racing(s,t,postponed) \neq \emptyset$).
  \item Report the real race, and randomly resolve it, which consists of
    executing one of the statements (and keep postponing the other) to check
    if something bad can happen.
    %
    The execution of one random statement will also generate the next state of
    the program for the continuation of the algorithm ($s := Execute(s,t)$).
  \end{enumerate}
\end{itemize}

\noindent
Particular cases:

\begin{itemize}
\item It may happen that all threads are postponed because, even though they
  are trying to execute a racing statement, none of them actually race as they
  access different shared memory locations.
  %
  The racing pair are the result of an imprecise hybrid race detection
  technique, which may report pairs that do not result in an actual race.
  %
  This situation turns into a deadlock in the \rfuz algorithm, that the
  scheduler breaks randomly by picking one of the threads from the $postponed$
  set.
\item \rfuz algorithm keeps looping until there are not $Enabled$
  threads. When it reaches this condition, if there are still active threads
  ($(Active(s) \neq \emptyset$) we have a deadlock in the concurrent program.
\end{itemize}

\subsection*{\rfuz case study}
\label{sec:member21}

The example in Figure~1 of~\cite{Sen:2008:RDR:1375581.1375584} is an
interesting case study which compares \rfuz with a pure
happens-before-based technique.
%
The program shows two threads that access three global variables $x, y, z$,
and according to a hybrid race detection technique, they race in two cases: on
variable $x$ and on variable $z$.
%
\rfuz will be invoked several times with different random seeds (to generate
different thread schedulings) on the $RaceSet=\{1, 10\}$ and on the
$RaceSet=\{5,7\}$.

\paragraph{Case 1:} $RaceSet=\{1, 10\}$. In case $thread1$ first reaches
statement $1$ it will be postponed.
%
Since $y == 0$, $thread2$ will never reach statement $10$ and will terminate,
therefore no real race will be reported.
%
The other schedule is likely to reach statement $10$ first, and as this is not
possible due to the initialization of the variables, no race will again be
reported.

\paragraph{Case 2:} $RaceSet=\{5, 7\}$. If $thread1$ reaches statement $5$
first, it is postponed until $thread2$ reaches statement $7$.
%
In this case \rfuz reports a real race since both statements access the same
memory location and one of them ($7$) is a write.
%
The same scenario happens if $thread2$ reaches first statement $7$.

\subparagraph{Discussion on Case1:} At the end of the first paragraph of
Section~2.2 of the paper, the author specifies that, even though the tool that
implements \rfuz uses hybrid race detection, any static or dynamic race
detection technique could be used.
%
The author claims that, in the Case~1 of the previous example, hybrid race
detection technique will always report a race over the variable $x$, however
this might not be true if the tool would use a pure happens-before relation.
%
In the example in Figure~1 of~\cite{Sen:2008:RDR:1375581.1375584} the only
schedule that would allow a hybrid technique to report the race on $x$ is the
one reported in Figure~\ref{fig:execution1}, where \emph{Thread~1} is executed
before \emph{Thread~2}, however this would not be the case for the
happens-before-based techniques.
%
All the other schedules would not allow \emph{Thread~2} to reach variable $x$
because of $y \neq 1$, so no technique would report a race.

Figure~\ref{fig:omprace1} shows indeed how the happens-before relation, based
on the schedule in Table~\ref{fig:execution1}, would not report the race on
$x$.
%
(Note that in this case I am not considering $z$ since it does not play an
important role).
%
Let us use DJIT+~\cite{Pozniansky:2007:MEO:1228965.1228969, Flanagan:2009} as
the algorithm that implements the happens-before in Figure~\ref{fig:omprace1}.
%
When $T2$ performs the last read on variable $x$, DJIT+ compares $T2$'s Vector
Clock (VC) with the write's VC for $x$.
%
DJIT+ applies the rule in Algorithm~\ref{alg:djit}.
%
$\mathbb{W}_x(t)$ ($<1, 0>$) and $\mathbb{C}_t(t)$ ($<1, 1>$) have the same
clock for \emph{Thread~1} and the clock for \emph{Thread~2} is smaller in the
write's vector clock of $x$, this satisfy the previous rule and no race is
reported.
%
In this particular case \rfuz does not perform better than an
happens-before-based technique, since Phase~1 alone would be able to give an
exact answer of the races in the program~\footnote{The race in Case~2 would be
  reported by both hybrid technique or happens-before alone.}.
\\

\noindent
Instead, the second example in Figure~2
of~\cite{Sen:2008:RDR:1375581.1375584}, shows a better scenario where the
happens-before relation would miss the race on variable $x$ with a default or
a simple random scheduler.
%
\rfuz in this case would allow to explore several schedules and create the
right scenario to detect and resolve the race condition.
%
Indeed, in Phase~1 the hybrid race detection will detect the pair of racing
statements $(8, 10)$.
%
In Phase~2, \rfuz will randomly pick one of the threads: if it picks $thread1$
the algorithm will first postpone the statement $8$, otherwise if it picks
$thread2$ the statement $10$ will be postponed.
%
In either case, \rfuz will report the race.

\begin{figure}
  \begin{minipage}[b]{0.28\textwidth}
    \centering
    \begin{algorithm}[H]
      \caption{DJIT+ Read rule}
      \label{alg:djit}
      \begin{algorithmic}
        \If {$\mathbb{R}_x(t) \neq \mathbb{C}_t(t)$}

        \texttt{assert} $\mathbb{W}_x \sqsubseteq \mathbb{C}_t$

        $\mathbb{R}_x(t) := \mathbb{C}_t(t)$
        \EndIf
      \end{algorithmic}
    \end{algorithm}
    \phantom{1}
    \phantom{2}
    \phantom{3}
    \phantom{4}
    \begin{tabular}{c c}
      Thread~1 & Thread~2 \\
      \hline
      w(x)   & \\
      acq(L) & \\
      w(y)   & \\
      rel(L) & \\
               & acq(L) \\
               & r(y) \\
               & r(x) \\
               & rel(L) \\
      \hline
    \end{tabular}
    \caption{Events Trace of example in Figure~1 in~\cite{Sen:2008:RDR:1375581.1375584}}
    \label{fig:execution1}
  \end{minipage}
  \hfill
  \begin{minipage}[b]{0.7\textwidth}
    \framebox[1.0\width]{
      \resizebox{1.0\textwidth}{!}{
        \xymatrix@C=0.5em@R=1.2em{
          \mathbf{T1} & & \mathbf{T2} & & \mathbf{L_L} & & \mathbf{R_x} & & \mathbf{W_x} & & \mathbf{R_y} & & \mathbf{W_y} \\
          <1, 0> \ar[d]^{w(x)} & & <0, 1> & & <0, 0> & & <0, 0> & & <0, 0> & & <0, 0> & & <0, 0> \\
          <1, 0> \ar[d]^{acq(L)} & & <0, 1> & & <0, 0> \ar@{-->}[lllld] & & <0, 0> & & <1, 0> & & <0, 0> & & <0, 0> \\
          <1, 0> \ar[d]^{w(y)} & & <0, 1> & & <0, 0> & & <0, 0> & & <1, 0> & & <0, 0> & & <0, 0> \\
          <1, 0> \ar[d]^{rel(L)} \ar@{-->}[rrrrd] & & <0, 1>  & & <0, 0> & & <0, 0> & & <1, 0> & & <0, 0> & & <1, 0> \\
          <2, 0> & & <0, 1> \ar[d]^{acq(L)} & & <1, 0> \ar@{-->}[lld] & & <0, 0> & & <1, 0> & & <0, 0> & & <1, 0> \\
          <2, 0> & & <1, 1> \ar[d]^{r(y)} & & <1, 0> & & <0, 0> & & <1, 0> & & <0, 0> & & <1, 0> \\
          <2, 0> & & <1, 1> \ar[d]^{r(x)} & & <1, 0> & & <0, 0> & & <1, 0> \ar@{-->}[lllllddd] & & <0, 1> & & <1, 0> \\
          & & <1, 1> \ar@{-->}[rdd] & & & & & & & & & & \\
          \\
          & & & W_x \sqsubseteq C_t \ar[d] & & & & & & & & &\\
          & & & \textbf{NO RACE} & & & & & & & & & }}
    }
    \caption{Race detection applying
      DJIT+~\cite{Pozniansky:2007:MEO:1228965.1228969} algorithm based on
      happens-before relation and vector clock, on example in Figure~\ref{fig:execution1}.}
    \label{fig:omprace1}
  \end{minipage}
  \vspace{-10pt}
\end{figure}

% \subsubsection*{Example 2}
% The example in Figure~2 of~\cite{Sen:2008:RDR:1375581.1375584} shows
% another example of two threads that access a global variable $x$.
% %
% The access on $x$ by the two threads will clearly results in a data race
% because of lack of synchronization.
% %
% Indeed, a hybrid data race detection technique will predict the race between
% statement $8$ and $10$.
% %
% However, because of the structure of the program, a happens-before-based
% detection techniques may not be able to report the race with high probability.
% %
% Indeed, this probability would depend on the scheduling of the two threads, a
% simple randomized scheduler would not execute the racing statement temporally
% next to each other and the happens-before relation would miss this race.
% %
% This example

\subsection*{Pros and Cons of \rfuz}
\label{sec:member22}

\rfuz differs from classical race detection techniques as it is able to
distinguish whether a race is harmful (or not), by executing potential racing
statements.
%
This is an important feature when used to distinguish real races from false
alarms reported by hybrid techniques.
%
However, the author claims that an advantage of \rfuz is that it separates
harmful races from benign races.
%
In my opinion, the benign races should still be reported as harmful races.
%
Whilst \rfuz may identify a race as benign, it is important to keep in mind
that a ``benign'' race in a debugging environment become harmful in production
because of compilers optimization, often omitted in the development phase.
%
As studied by Boehm~\cite{Boehm:2011:MPB:2001252.2001255}, compilers assume
race free code and apply transformations that, in presence of a ``benign''
race, might results in harmful races~\cite{ec2_2015-agralslpm}.
%
\rfuz also identifies those races that do not generate any program exception as
``benign''.
%
This is a major downside of the algorithm, since this category includes the
races that might produce wrong results (i.e.\ bad non-determinism~\footnote{I
  call it ``bad non-determinism'' to distinguish it from other types of
  non-determinism such as the one caused by the order in floating-point
  operations that it is in general expected and sometimes unavoidable.}).
%
It is not clear if \rfuz, when it terminates the execution, will include in
the report all the real races, or only those ones that lead to a program
exception or error.

A nice feature of \rfuz is that the random scheduler is regulated by a random
number generator, it means that \rfuz can replay the error scenario, simply
using the same seed (no events recording needed, lightweight replay
mechanism).
%
Furthermore, \rfuz is highly parallelizable.
%
Each execution of the algorithm is independent, since it only focuses on one
pair of racing statements at a time.
%
Therefore, it allows the user to run multiple instances of \rfuz in parallel,
increasing the performance linearly with the number of available
processors/cores.
%
Similar to all dynamic race detection techniques, \rfuz is not able to detect
all races in a concurrent program, but only those races that can be produced
by a given tests.
%
It might not be able to distinguish all harmful races from potential races,
because the random scheduler might not generate the right schedule to
reproduce the real race.

As shown in the experimental results, \rfuz introduces additional overhead to
the entire race detection process.
%
However, the overhead is negligible as the majority arise from the data race
detection technique used.
%
The results also show that the algorithm is very precise, any races found in
the benchmarks were confirmed, and the technique guarantees to recreate a data
race scenario with high probability.

\vspace{-10pt}
\paragraph{\rfuz on OpenMP programs:}
The \rfuz can be implemented for any language that supports shared memory and
threads programming model (i.e.\ C, C++ or Java).
%
Considering languages as C and C++, such a tool, would allow to detect races
not only on PThreads applications, but also on OpenMP programs since they are
also based on PThreads.
%
Existing tools, such as \archer~\cite{Protze:2014:TPL:2688361.2688369},
already do a good job in finding races on OpenMP applications.
%
Indeed, experiments show that \archer does not report any false alarms, and so
far there have not been any cases where it misses races (with regard to known
races).

However, \archer is based on a tool called ThreadSanitizer, which rely on the
happens-before relation.
%
As illustrated in the second example of the paper, there are cases where
happens-before might miss data races making a tool like \archer imprecise.
%
The \rfuz approach would mitigate this type of problem, by running the program
with different possible interleavings that would manifest the latent races.
%
With regards to this, it is important to clarify the cases which \rfuz would
improve the precision of a tool like \archer.
%
OpenMP provides several constructs to automatically parallelize C, C++
programs.
%
In general, when a programmer uses construct such as parallel loops, follows a
very precise structure, that allow to write parallel loop with
no-synchronization mechanism as shown in Figure~\ref{fig:openmp1}.
%
Moreover, branches within parallel loop are very rare.
%
In the case of data races in those situations, as illustrated in
Figure~\ref{fig:openmp2}, it is unusual that the threads scheduling could lead
a tool like \archer to miss a race.
%
Instead, other OpenMP constructs such as tasks, parallel sections, and
branches within them, may increase the probability to lead an
happens-before-based tool to miss certain races.
%
An example of this situation could be the program in Figure~2 of the paper,
where the code of $thread1$ and $thread2$ (when translated to OpenMP) is
respectively executed within two different OpenMP task constructs by two
threads, as illustrated in Figure~\ref{fig:openmp3}.
%
In the OpenMP version, as well as in the PThreads one, a happens-before-based
detection will not be able to detect the race with high probability, because
the threads scheduling might not manifest the right order to catch it.

\begin{figure}[t]
  \begin{minipage}[t]{0.4\textwidth}
\begin{lstlisting}[language=C]
  #pragma omp parallel for
  for (i = 0; i < N; i++)
     a[i] = a[i] + 1;
\end{lstlisting}
    \vspace{-10pt}
\caption{OpenMP parallel for that does not need any synchronization mechanism
  to be race free.}
\label{fig:openmp1}
\end{minipage}
\hfill
\begin{minipage}[t]{0.4\textwidth}
\begin{lstlisting}[language=C]
  #pragma omp parallel for
  for (i = 1; i < N; i++)
     a[i] = a[i - 1];
\end{lstlisting}
  \vspace{-10pt}
\caption{OpenMP parallel for with potential data race because of a data
  dependency in the access of array $a$.}
\label{fig:openmp2}
\end{minipage}
\vspace{-10pt}
\end{figure}

On the other hand, if it is true that \rfuz would report only those data races
that lead to an actual error or an exception in the program, this technique
would not be very useful, since it would miss to report actual races that are
not classified as harmful.
%
My research focuses on data race detection for OpenMP programs, and the main
goal is to design better techniques to detect any kind of data race, even
though they are considered non-harmful or ``benign''.
%
I specifically apply my data race detection research on HPC scientific
programs, where previous research~\cite{ec2_2015-agralslpm} has shown that any
kind of data race could be critical, even if it is classified as ``benign''.
%
I believe that the \rfuz approach, after a customization to make it report all
races any type of actual races, may prove useful to improve the data race
detection for OpenMP applications.

Another important goal in my research is to reduce the considerable amount of
overhead generated by race detector tools, as when they are used to check
large OpenMP applications (as in HPC), the slowdown is even larger than when
they are applied to regular OpenMP programs.
%
% \rfuz has shown very low overhead in the second phase of the algorithm where
% it tries to reproduce possible racy interleavings.
%
% Given that, and the fact \rfuz can use any race detection technique for the
% first phase, a possible approach that may reduce the overhead of data race
% detection is to combine \rfuz with a static race detection technique.
%
Given that \rfuz can use any race detection technique for the first phase, and
the fact it has shown very low overhead in the second phase, a possible
approach that may reduce the overhead of data race detection is to combine
\rfuz with a static technique.
%
Static race detection techniques are known to be very scalable but imprecise,
i.e.\ they reports many false alarms.
%
However, the second phase of \rfuz can be used to verify if the data races
reported by the static technique are actual races or not.
%
Hypothetically, this combination could reduce significantly the overhead of
the race detection process while maintaining high precision and accuracy.

\begin{figure}[H]
  \vspace{-10pt}
  \begin{minipage}[t]{0.35\textwidth}
    \begin{lstlisting}[language=C, xleftmargin=-30pt, xrightmargin=0pt]
      |\textbf{main:}|
      #pragma omp parallel
      {
        #pragma omp task shared(x)
        thread1();

        #pragma omp task shared(x)
        thread2();
      }
    \end{lstlisting}
  \end{minipage}
  \hfill
  \begin{minipage}[t]{0.3\textwidth}
    \begin{lstlisting}[language=C, xleftmargin=-15pt, xrightmargin=0pt]
      |\textbf{thread1:}|
      #pragma omp critical
      {
        f1();
        |\vdots|
        f5();
      }
      if(x == 0)
        ERROR;
    \end{lstlisting}
  \end{minipage}
  \hfill
  \begin{minipage}[t]{0.3\textwidth}
    \begin{lstlisting}[language=C, xleftmargin=-30pt, xrightmargin=0pt]
      |\textbf{thread2:}|
      x = 1;
      #pragma omp critical
      {
        f6();
      }
    \end{lstlisting}
  \end{minipage}
  \vspace{-10pt}
  \caption{Example of Figure~2 of the paper implemented with OpenMP tasks.}
  \label{fig:openmp3}
  \vspace{-15pt}
\end{figure}

\printbibliography

\end{refsection}

%%% Local Variables:
%%% mode: latex
%%% eval: (flyspell-mode 1)
%%% TeX-master: "root.tex"
%%% End:
